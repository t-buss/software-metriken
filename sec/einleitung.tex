\section{Einleitung}
In diesem Abschnitt soll einen Einstieg in die Thematik gegeben werden.
Es wird zunächst auf die Problemstellung eingegangen und eine Motivation für das Forschungsthema geben.
Danach folgt die Formulierung eines praxisorientierten Ziels für die weiteren Überlegungen.

\subsection{Problemstellung}
Die heutzutage als selbstverständlich geltende Allgegenwärtigkeit von Software-gesteuerten Maschinen geht mit einem hohen Gefahrenpotential einher.
Software ist in fast jedem Aspekt unseres täglichen Lebens involviert.
Fehler oder Sicherheitslücken können daher weitreichende Folgen haben, die von Datenschutzverletzungen über finanziellen Schaden bis zur Lebensgefahr reichen.

Es ist zweifelsohne wünschenswert, derartige Fehler in Software so schnell wie möglich zu erkennen und zu beheben.
Gerade bei sicherheitsbezogenen Problemen ist die Lösung des Problems jedoch schwer:
Sicherheitsexperten, die Schwachstellen in der Software erkennen und beheben können, sind selten und vergleichsweise teuer.
Auch wenn ein solcher Experte verfügbar ist, so würde die Überprüfung eine große Zeitspanne einnehmen und daher nicht praktikabel.

Ein automatisches System zur Erkennung von potentiellen Schwachstellen wäre wünschenswert.
Dieses könnte dabei assistieren, die Arbeit von Sicherheitsexperten auf die Code-Bereiche zu fokussieren, die relevant für eine Überprüfung sind.
Dadurch könnte viel Arbeitszeit und somit auch Kosten reduziert werden.

Es stellt sich die Frage, mit welchen Mitteln ein solches System Schwachstellen erkennen könnte.
Software-Metriken stellen in diesem Kontext einen geeigneten Kandidaten dar.
Sie werden hauptsächlich für die Erfassung von Software-Qualitätsmaßen verwendet, können aber auch, wie im Verlauf der Ausarbeitung gezeigt wird, verwendet werden, um Schwachstellen aufzudecken.

\subsection{Zielsetzung}
