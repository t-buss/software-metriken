\section{Metriken}
\label{sec:metriken}
Zunächst muss der Begriff "`Metrik"' erläutert werden, da dieser zentral für den Rest der Ausarbeitung ist.
Dieser Abschnitt beschäftigt sich mit den Arten von Metriken, die für Code-Analysen in den Beispielstudien in Abschnitt \ref{sec:studien} verwendet werden.
Er hat keinen Anspruch auf Vollständigkeit; andere Metriken für die Analyse sind denkbar.

\subsection{Ebenen der Software-Metrik}
Software-Metriken können auf unterschiedlichen Ebenen erfasst werden, beispielsweise auf Datei-, Klassen- oder Funktionsebene.
Die Wahl der entsprechenden Ebene kann von Faktoren wie der Programmiersprache oder auch der Architektur der Anwendung abhängen.
Viele Programmiersprachen haben kein Klassenkonzept.
Dadurch können viele Metriken, die für objektorientierte Programmierung gedacht sind, nicht verwendet werden und es erschwert den Vergleich mit Projekten in objektorientierten Sprachen.
Datei- und Funktionsebene sind in diesem Kontext "`universeller"', da sie in allen gängigen Anwendungen vorhanden sind.

\subsection{Anzahl Codezeilen (LoC)}
Ein einfaches Beispiel für eine Software-Metrik ist die Anzahl der Codezeilen.
Komponenten von hoher Software-Qualität haben weniger Codezeilen, was sie übersichtlicher und wartbarer machen.
Komponenten mit vielen Codezeilen versuchen, mehrere Aufgaben zu erledigen.
Durch eine bessere Aufteilung in kleinere Komponenten wird die Software einfacher zu handhaben und ist weniger fehleranfällig.

\subsection{McCabe's Cyclomatic Complexity}
Die zyklomatische Komplexität oder auch \emph{McCabe-Metrik}~\cite{mccabe1976} misst die Anzahl der unterschiedlichen Ausführungspfade innerhalb eines Codeabschnittes.
Hat ein Abschnitt keinerlei Fallunterscheidungen oder Schleifen, so ist dessen zyklomatische Komplexität 1.
Gibt es eine Fallunterscheidung, so gibt es zwei Pfade, die das Programm ablaufen kann.
Die zyklomatische Komplexität ist demnach 2.
Die Formel zur Berechnung der zyklomatischen Komplexität anhand eines Flussgraphen ist
\begin{equation}
	M = E - N + 2P
\end{equation}
wobei E die Anzahl der Kanten im Flussgraphen, N die Anzahl der Knoten und die Anzahl der Zusammenhangskomponenten ist.
Eine Zusammenhangskomponente entspricht dabei einer Funktion oder Prozedur.

Für diese Metrik gibt es Variationen, bei denen die Anzahl der unterscheidbaren Pfaden in Flussgraphen unterschiedlich berechnet wird.
