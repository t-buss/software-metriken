\section{Hypothesen}
\label{sec:hypothesen}
Nachdem die Begrifflichkeiten erläutert wurden, lassen sich mit deren Hilfe Hypothesen formulieren, die den Rest dieser Ausarbeitung als Leitfaden dienen sollen.
Sicherheitslücken treten auf, wenn Entwickler aufgrund der Komplexität des Systems unabsichtlich defekte Komponenten entwickeln.
Es stellt sich also die Frage, ob diese Schwächen in der Software-Architektur,
sowohl auf einer höheren, abstrakten Ebene, als auch auf einer niedrigeren, konkreten Ebene,
sich durch Metriken erkennen und somit potentiell defekte Komponenten mit Sicherheitslücken aufspüren lassen.

Viele Hypothesen sind für dieses Forschungsgebiet denkbar.
Die nachfolgenden Hypothesen sollen in dieser Ausarbeitung näher betrachtet werden.
Sie orientieren sich an der Forschung von Alves et. al (siehe \cite{alves_et_al}).

\subsection{Vorhersagbarkeit von Sicherheitslücken durch Metriken}
Die erste Hypothese lautet: "`Durch Software-Metriken lassen sich Rückschlüsse auf die Existenz von Sicherheitslücken schließen"'.
Dies ist die zentrale Hypothese der Ausarbeitung.
Wenn sie bestätigt werden kann, lassen sich durch Software-Metriken automatisierte Werkzeuge für Entwickler erstellen, die bei der Auffindung von Sicherheitslücken helfen.

\subsection{Vorhersagbarkeit von Anzahl an Sicherheitslücken}
Die zweite Hypothese lautet: "`Durch Software-Metriken lassen sich Rückschlüsse auf die Anzahl der Sicherheitslücken schließen"'.
Wenn sich die erste Hypothese bestätigen lässt, ist es zusätzlich von Vorteil, die Anzahl der Sicherheitslücken voraussagen zu können.

\subsection{Korrelation zwischen Software-Metriken}
Die dritte Hypothese lautet: "`Einige Software-Metriken korrelieren miteinander"'.
Es ist nicht bekannt, welche Software-Metriken sich besonders gut oder schlecht für die Vorhersage von Sicherheitslücken eignen.
Außerdem könnten sich Redundanzen und Widersprüche in den Vorhersagen finden.
Daher ist es naheliegend zu untersuchen, welche Menge von Software-Metriken für eine Vorhersage ausgewählt werden sollte.
