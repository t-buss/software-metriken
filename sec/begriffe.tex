\section{Begriffe}
\label{sec:begriffe}
Bevor sich Hypothesen über den Zusammenhang zwischen Software-Metriken und Sicherheitslücken formulieren lassen, müssen diese beiden Begriffe zunächst erläutert werden.

\subsection{Software-Metrik}
Software-Metriken beschreiben Charakteristika eines Software-Systems mittels eines Zahlenwertes.
Speziell geht es in dieser Ausarbeitung um Metriken für ein einzelnes, abgeschlossenes Programm.
Neben diesen gibt es im sicherheitsbezogenen Kontext auch Metriken, die in einem Rechnernetzwerk erfasst werden können~\cite{cheng2014}.
Zur Erfassung von Software-Metriken können statische Code-Analysewerkzeuge verwendet werden.
Software-Metriken lassen sich auf verschiedene Art und Weise in Gruppen einteilen.
Eine mögliche Gruppierung ist die Einteilung nach Code-Level- und Design-Level-Metriken.
Unter Design-Level-Metriken versteht man solche, die nach dem Entwurf der Software (beispielsweise anhand eines UML-Klassendiagramms) berechnet werden können.
Implementierungsspezifische Metriken, wie der Anzahl Code-Zeilen, bezeichnen sich als Code-Level-Metriken.
Eine andere Gruppierung ist die Einteilung nach Komplexität-, Kopplungs- und Kohäsionsmetriken.
Diese Metriken bezeichnet man auch als CCC-Metriken und werden in den folgenden Abschnitten genauer erläutert.

\subsubsection{Komplexität}
Komplexitätsmaße beschreiben, inwieweit die Komponenten eines Systems miteinander "`verflochten"' sind.
Den Begriff zu definieren fällt dabei schwer.
Es lassen sich aber Maße finden, die Eigenschaften komplexer Systeme ausdrücken.
Ein Beispiel eines solchen Maßes ist \textit{McCabe's Cyclomatic Complexity}~\cite{mccabe1976}.
Sie zählt die Anzahl der unterschiedlichen Kontrollflüsse innerhalb eines Programms.

Komplexitätsmaße als Indikatoren für Fehler zu verwenden ergibt Sinn.
Je einfacher das System zu verstehen ist, desto weniger fehleranfällig ist es.
Fehler in komplexen Systemen zu finden gestaltet sich hingegen als schwierig.

\subsubsection{Kopplung}
Kopplungsmaße beschreiben, wie sehr einzelne Komponenten voneinander abhängig sind.
Hohe Kopplung bedeutet, dass eine Änderung viele Änderungen als Folge nach sich zieht.
Der betroffene Code sollte so klein wie möglich gehalten werden.
Ein Beispiel für Kopplungsmaße ist die Anzahl der Unterklassen oder "`Number of Children (NOC)"'.
Durch Vererbung wird die Unterklasse an die Oberklasse gekoppelt.

Kopplungsmaße sind ein guter Indikator für die Wartbarkeit von Software-Systemen.
Eine Betrachtung von Kopplung für die Findung von Fehlern ergibt daher Sinn.

\subsubsection{Kohäsion}
Kohäsion beschreibt, inwieweit die Methoden und Variablen einer Klasse zueinander in Verbindung stehen.
So sollte sich eine Klasse nach dem "`Single Responsibility Principle"' nur um eine Aufgabe oder ein Konzept kümmern.
Geringe Kohäsion ist ein Anzeichen dafür, dass eine Klasse in mehrere aufgeteilt werden sollte.
Hohe Kohäsion bewirkt, dass die Klasse einfacher zu verstehen ist und erhöht die Wiederverwendbarkeit.
Ein Beispiel für ein Kohäsionsmaß ist "`Lack of cohesion of methods (LCOM)"', in dem die Anzahl der Variablen, Methoden und Methoden, die diese Variablen verwenden, miteinander verrechnet werden.

\subsection{Sicherheitslücke}
Der Begriff \emph{Sicherheit} bezeichnet mehrere Eigenschaften.
Oftmals wird bei Sicherheit nur an Vertraulichkeit gedacht, also die Verhinderung des Zugriffs auf vertraulichen Informationen durch Unbefugte.
Aber auch die Integrität der Daten und die Verfügbarkeit des Systems gehören dazu.

Eine \emph{Sicherheitslücke} ist vorhanden, wenn ein interner Defekt einer Komponente einem externen Fehlerursache die Möglichkeit bietet, eine oder mehrere Eigenschaften der Sicherheit zu verletzen~\cite{basics}.
Die externe Fehlerursache kann sich sowohl auf fehlerhafte Komponenten als auch unbefugte Personen beziehen.

Sicherheitslücken treten immer wieder in Software-Systemen auf.
Sie entstehen durch menschliche Fehler während der Entwurfs-, Entwicklungs- und Deployment-Phasen eines Systems, insbesondere, wenn eine ungünstige Architektur die Verständlichkeit einer Komponente erschwert.
Eine allumfassende Planung zum Ausschluss von Sicherheitslücken ist praktisch unmöglich.
Daher muss neben der vorangehenden Planung eines System auch während der Entwicklung überprüft werden, ob Sicherheitslücken vorliegen.
