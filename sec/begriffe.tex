\section{Begriffe}
\label{sec:begriffe}
Bevor sich Hypothesen über den Zusammenhang zwischen Software-Metriken und Sicherheitslücken formulieren lassen, müssen diese beiden Begriffe zunächst erläutert werden.
%TODO Unterschied zu Netzwerk-Metriken zur Sicherheit
\subsection{Software-Metrik}

\subsubsection{Complexity}
% Komplexität
% Große Zahl an Komponenten die miteinander interagieren, lokales Verhalten, sodass auf höherer Ebene kein Überblick herrscht
% Definition schwierig/unmöglich, aber Charakterisierung möglich
% Eine Möglichkeit: McCabe's Complexity: Anzahl unterschiedlicher Pfade innerhalb eines Programms
% Komplexität != Kompliziertheit: Kompliziert heißt viele Schichten
% Motivation: Das Gesamtsystem soll einfach zu verstehen sein

\subsubsection{Coupling}
% Kopplung
% Wie abhängig sind Komponenten voneinander? Besser simples Interface als vollständig abhängig von der Implementierung zu sein
% Motivation: Code soll einfach zu ändern sein, betroffener Bereich einer Änderung möglichst klein halten

\subsubsection{Cohesion}
% Kohäsion
% Inwiefern "passen" die Methoden und Variablen einer Klasse zueinander? Gibt es viele Methoden, die geteilte Variablen nicht benutzen? -> schwache Cohesion
% In dem Fall sollte Klasse in zwei oder mehr Klassen aufgeteilt werden
% Motivation: Single Resposibilty Principle

\subsection{Sicherheitslücke}
% Dienst (Service) (von außen wahrgenommenes Verhalten einer Komponente) ist korrekt -> Dienst erfüllt seine Funktion

% Failure ~= Fehlschlag: Ereignis
% Error ~= Fehler: Abweichung vom definierten Verhalten
% Fault ~= Defekt: Grund für Abweichung

% Sicherheit: nicht nur Vertraulichkeit (keine nicht-autorisierte Weitergabe von vertraulichen Informationen), sondern Integrität und Verfügbarkeit

% Vulnerability ~= Sicherheitslücke, verursacht durch einen internen Defekt (vault), der externen Defekt (Bot oder "Hacker") Schaden verursachen lassen kann
