\section{Begriffe}
\label{sec:begriffe}
Bevor sich Hypothesen über den Zusammenhang zwischen Software-Metriken und Sicherheitslücken formulieren lassen, müssen diese beiden Begriffe zunächst erläutert werden.
%TODO Unterschied zu Netzwerk-Metriken zur Sicherheit
\subsection{Software-Metrik}

\subsubsection{Komplexität}
Komplexitätsmaße beschreiben, inwieweit die Komponenten eines Systems miteinander "`verflochten"' sind
\footnote{Das Wort "`Komplexität"' stammt aus dem Lateinischen und bedeutet "`mit verflochten"'}.
Eine hohe Komplexität verursacht, dass man sich von einer höheren Abstraktionsebene keinen Überblick über das System machen kann
\footnote{Komplex ist nicht das Gleiche wie kompliziert. Komplizierte Systeme haben viele Abstraktionsebenen, die den Überblick erschweren. Einfach ausgedrückt bezieht sich "`komplex"' auf Verflechtungen und "`kompliziert"' auf Schichten.}.
Den Begriff zu definieren fällt dabei schwer.
Es lassen sich aber Maße finden, die Eigenschaften komplexer Systeme ausdrücken.
Ein Beispiel eines solchen Maßes ist McCabe's Cyclomatic Complexity~\cite{mccabe1976}.
Sie zählt die Anzahl der unterschiedlichen Kontrollflüsse innerhalb eines Programms.

Komplexitätsmaße als Indikatoren für Fehler zu verwenden ergibt Sinn.
Je einfacher das System zu verstehen ist, desto weniger fehleranfällig ist es.
Fehler in komplexen Systemen zu finden gestaltet sich hingegen als sehr schwierig.

\subsubsection{Kopplung}
Kopplungsmaße beschreiben, wie sehr einzelne Komponenten voneinander abhängig sind.
Hohe Kopplung ist negativ zu betrachten, da in einem solchen Fall eine Änderung der einen Komponente eine Änderung in einer anderen Komponente bedingt.
Der betroffene Code sollte so klein wie möglich gehalten werden.
Ein Beispiel für Kopplungsmaße ist die Anzahl der Unterklassen oder "`Number of Children (LOC)"'.
Durch Vererbung wird die Unterklasse an die Oberklasse gekoppelt.

Kopplungsmaße sind ein guter Indikator für die Wartbarkeit von Software-Systemen.
Eine Betrachtung von Kopplung für die Findung von Fehlern ergibt daher Sinn.

\subsubsection{Kohäsion}
Kohäsion beschreibt, inwieweit die Methoden und Variablen einer Klasse zueinander in Verbindung stehen.
Eine hohe Kohäsion bewirkt, dass der Programmierer nicht zu viele Dinge gleichzeitig beachten muss.
So sollte sich eine Klasse nach dem "`Single Responsibility Principle"' nur um eine Aufgabe oder ein Konzept kümmern.
Geringe Kohäsion ist ein Anzeichen dafür, dass eine Klasse in mehrere aufgeteilt werden sollte.

\subsection{Sicherheitslücke}
% Dienst (Service) (von außen wahrgenommenes Verhalten einer Komponente) ist korrekt -> Dienst erfüllt seine Funktion

% Failure ~= Fehlschlag: Ereignis
% Error ~= Fehler: Abweichung vom definierten Verhalten
% Fault ~= Defekt: Grund für Abweichung

% Sicherheit: nicht nur Vertraulichkeit (keine nicht-autorisierte Weitergabe von vertraulichen Informationen), sondern Integrität und Verfügbarkeit

% Vulnerability ~= Sicherheitslücke, verursacht durch einen internen Defekt (vault), der externen Defekt (Bot oder "Hacker") Schaden verursachen lassen kann
