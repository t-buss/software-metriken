\section{Beispielstudien}
\label{sec:studien}
Um einen Einblick in das Forschungsthema zu geben, werden in diesem Abschnitt zwei Studien beispielhaft vorgestellt, in denen untersucht wird, inwieweit sich Software-Metriken für die frühe Erkennung von Sicherheitslücken eignen.


\subsection{Alves et. al}

\subsection{Chowdhury und Zulkernine}

\subsection{Andere Forschungsarbeiten}
Neben den zuvor vorgestellten Forschungsarbeiten von Alves et. al und Zulkernine sind noch weitere Arbeiten zu erwähnen, die jedoch nicht den Schwerpunkt dieser Ausarbeitung bilden.

\subsubsection{Metrics of Security, Cheng et. al}
In dem Kapitel "`Metrics of Security"' von Yi Cheng et. al~\cite{cheng2014} aus dem Buch "`Cyber Defense and Situational Awareness"' geht es um Metriken für die Messung von Sicherheit in Netzwerken.
Die Autoren verwenden dafür den Begriff "`Cyber Situational Awareness"' um zu beschreiben, wie bewusst sich die Administratoren eines Netzwerkes über den Sicherheitsstatus sind.
Um die Verbesserung durch Maßnahmen quantitativ messen zu können, werden in diesem Kapitel Metriken für die Sicherheit definiert.
