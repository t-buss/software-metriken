\section{Ausblick}
\label{sec:ausblick}
Zum Schluss dieser Ausarbeitung soll noch ein Ausblick über künftige Entwicklungen auf diesem Forschungsgebiet gegeben werden.

Die Verfahren zur Vorhersage können noch weiter optimiert werden, um bessere Vorhersagemodelle zu entwickeln.
Dazu gehören neue Analysemethoden zur Entdeckung von Korrelationen als auch bessere Modelle für Klassifikatoren.
Dadurch könnten noch unbekannte Zusammenhänge erkannt werden.

Die Eignung von anderen Metriken für die Vorhersage von Sicherheitslücken kann weiter untersucht werden.
Bisherige Untersuchungen haben sich auf objekt-orientierte Programmiersprachen die C++ konzentriert.
Eine Ausweitung auf andere Programmiersprachen und Software-Architekturen wie service-orientierte Architekturen könnte weitere Erkenntnisse liefern~\cite{chowdhury_zulkernine_2009}.

Ein Problem, welches sich bei Alves et al. gezeigt hat, ist die Datenerhebung~\cite{alves_et_al}.
Viele Projekte haben keine Security Advisories und bei denen, die diese verwenden, konnten nicht alle Einträge verarbeitet werden.
Ein besseres Verfahren zur Datengewinnung könnte die Präzision der Ergebnisse verbessern und repräsentativer machen.
