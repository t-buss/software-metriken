\section{Ausblick}
\label{sec:ausblick}
In den vorangegangenen Abschnitten wurden die aktuellen Forschungsergebnisse zu diesem Gebiet vorgetragen und zusammengefasst.
Zum Schluss dieser Ausarbeitung soll noch ein Ausblick über künftige Entwicklungen auf diesem Forschungsgebiet gegeben werden.

Die Verfahren zur Vorhersage und die Auswahl der verwendeten Metriken kann noch weiter optimiert werden, um bessere Vorhersagemodelle zu entwickeln.

Die Eignung von anderen Metriken für die Vorhersage von Sicherheitslücken kann weiter untersucht werden.
Bisherige Untersuchungen haben sich auf objekt-orientierte Programmiersprachen die C++ konzentriert.
Eine Ausweitung auf andere Programmiersprachen und Software-Architekturen wie service-orientierte Architekturen könnte weitere Erkenntnisse liefern\cite{chowdhury_zulkernine_2009}.

Ein Problem, welches sich bei Alves \emph{et al} \cite{alves_et_al} gezeigt hat, ist die Datenerhebung.
Lediglich 55\% der Sicherheitslücken konnten durch das verwendete Programm analysiert werden.
Zudem kommen nicht viele Projekte für die Analyse in Frage, da nur wenige Open Source Projekte Security Advisories haben\footnote{Daher haben auch viele der betrachteten Studien \emph{Mozilla Foundation Security Advisories} als Quelle der Daten verwendet}.
Ein besseres Verfahren zur Datengewinnung könnte die Präzision der Ergebnisse verbessern und repräsentativer machen.
