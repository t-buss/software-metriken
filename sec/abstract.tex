\begin{abstract}
Diese Ausarbeitung soll einen Überblick über das Forschungsthema der Erkennung von Sicherheitslücken in Quellcode"=Dateien anhand von Software-Metriken geben.
Ziel ist es, ein Indikations-Tool zu erstellen, welches auf potenziell fehlerhafte Stellen im Quellcode hinweist und damit Sicherheitsexperten bei ihrer Arbeit unterstützt.
Dazu werden zunächst Software-Metriken, wie die Anzahl der Pfade im Flussdiagramm einer Software oder die Kopplung zwischen verschiedenen Klassen, erfasst.
Daran knüpft eine Analyse auf statistische Signifikanz der erfassten Daten an.
Die Ergebnisse zeigen, dass Software-Metriken sich dazu eignen, Stellen mit Sicherheitslücken im Quellcode vorrauszusagen.
Zum Schluss wird ein Ausblick über die künftige Entwicklung des Themas gegeben.

\keywords{Security  \and Metrics \and Vulnerabilities \and Indication \and Prediction \and CCC \and Static Code Analysis}
\end{abstract}
