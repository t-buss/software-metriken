\section{Fazit}
\label{sec:fazit}
Es konnte gezeigt werden, dass Software-Metriken mit Sicherheitslücken korrelieren und sich damit Codestellen mit potenziellen Sicherheitslücken erfassen lassen.
Eine verlässliche Voraussage über Anzahl der Sicherheitslücken lässt sich nicht treffen.
Die Ergebnisse sind über mehrere Versionen der gleichen Software als auch bei verschiedenen Projekten konsistent.
Für eine optimale Analyse müssen die verwendeten Metriken projektabhängig gewählt werden und sollten verschiedene Ebenen (Funktions-, Datei- und Projektebene) abdecken.
Die historische Betrachtung der Sicherheitslücken eines Projektes kann die Ergebnisse verbessern.
Dieses Wissen kann genutzt werden, um automatisierte Tools zu erstellen, die bei der Entwicklung helfen und Schwachstellen frühzeitig erkennen.
